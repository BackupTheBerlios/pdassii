La idea inicial del proyecto, era desarrollar una aplicaci'on similar a la desarrollada por Miguel A. Mu'noz, Marcela Rodr'iguez, Jes'us Farela y Ana I. Mart'inez-Garc'ia, de Universidad de M'exico, en colaboraci'on con V'ictor M. Gonz'alez, de la Universidad de California, para entornos hospitalarios. Dicha aplicaci'on tiene como finalidad la mejora de las condiciones de trabajo del personal hospitalario introduciendo en el hospital la funcionalidad del PDA. As'i, el personal dotado de PDAs puede comunicarse con otros compa'neros de trabajo, incluso de otros turnos, y tener acceso a datos de sus pacientes con facilidad y rapidez. La aplicaci'on se prob'o en  el Hospital General de Ensenada, M'exico, con resultados satisfactorios.Nuestra aplicaci'on pretende semejarse a ella pero es un poco m'as limitada. \bigskip \\ La aplicaci'on que hemos desarrollado tambi'en est'a pensada para el entorno hospitalario: el PDA debe ser una herramienta de trabajo m'as para el m'edico. Las ventajas que puede aportar el uso de PDA a un m'edico son varias pero destaca la rapidez, en ocasiones, cr'itica en el entorno que nos movemos. El m'edico tiene acceso a informaci'on de uso frecuente y necesario sin tener que acudir a personal auxiliar del hospital, con el consiguiente ahorro de personal que puede suponer al hospital. Adem'as, el interfaz del PDA es muy sencillo y cualquier persona que est'e familiarizada con un tel'efono m'ovil puede entenderla.\bigskip \\ Durante este a'no nos hemos centrado en un caso de uso concreto: el poder proporcionar al m'edico usuario del PDA la posibilidad de imprimir documentos en la impresora m'as cercana. Esto es posible a trav'es del algoritmo de triangulaci'on implementado que no s'olo ser'ia 'util en este terreno, sino que ser'ia aplicable a otro casos de uso, como mostrar informaci'on por monitores, etc.\bigskip \\ Adem'as de esta facilidad, el m'edico tambi'en tiene la posibilidad de consultar el expediente del paciente al que est'a atendiendo, visualiz'andolo en su PDA.\bigskip \\ Las posibles ampliaciones de la aplicaci'on son m'ultiples y variadas. Adem'as, el dise'no modular con que la hemos llevado a cabo hace que nuevos casos de uso sean muy f'aciles de a'nadir. El hecho de haber escogido tecnolog'ias libres y multiplataforma tambi'en es una ventaja.\bigskip \\ La aplicaci'on tambi'en tiene la ventaja de que es exportable a otros entornos como pueden  ser museos, almacenes, etc.