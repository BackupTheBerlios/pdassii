Este es es diagrama de secuencia de la llamada a la localizaci'on (figura \ref{fig:localizador}).
\begin{figure*}[h!]
	\begin{center}
        		\framebox{\includegraphics[scale=0.6]{localizador.png}}
     	\end{center}
    	\caption{Secuencia de localizaci'on}
	\label{fig:localizador}
\end{figure*}

El proceso de localizaci'on consta de dos partes:
\begin{itemize}
\item Medir las distancias a los puntos de acceso.
\item aplicar el algoritmo de triangulaci'on.
\end{itemize}
Para medir las distancias hay que tener en cuenta la distribuci'on de los puntos de acceso formando cuadrados.
Pedimos al servidor los puntos que est'en m'as pr'oximos. Medimos la distancia al punto al que nos conectamos (punto A) con una llamada a ping. A continuaci'on hacemos ping al segundo punto de la lista (punto B). Como sabemos la distancia entre los dos puntos y la distancia de la PDA (punto C) al punto A, solo tenemos que aplicar reglas geom'etricas para conocer la distancia del punto B al C, ya que forman un tri'angulo. 
Repetimos este proceso para los otros dos puntos de la lista.\bigskip \\ La llamada a la funci'on triangula se encarga de la localizaci'on de la PA una vez que se tienen medidas las distancias a los puntos de acceso.\bigskip \\ La localizaci'on corresponde a la resoluci'on del sistema de ecuaciones que se define por el punto de intersecci'on de tres esferas con centro el las coordenadas del punto de acceso y radio la distancia al dispositivo inal'ambrico a localizar. Si los puntos de acceso son:

\begin{enumerate}
\item Punto 1: 
	\begin{enumerate}
	\item Coordenadas: $(x_1,y_1,z_1)$
	\item Distancia: $d_1$
	\end{enumerate}

\item Punto 2: 
	\begin{enumerate}
	\item Coordenadas: $(x_2,y_2,z_2)$
	\item Distancia: $d_2$
	\end{enumerate}

\item Punto 3: 
	\begin{enumerate}
	\item Coordenadas: $(x_3,y_3,z_3)$
	\item Distancia: $d_3$
	\end{enumerate}

\item Punto 4: 
	\begin{enumerate}
	\item Coordenadas: $(x_4,y_4,z_4)$
	\item Distancia: $d_4$
	\end{enumerate}
\end{enumerate}

Para localizar el punto ($x$,$y$,$z$) tenemos que resolver:
\begin{enumerate}
	\item ecuación 1: $(x - x_1)² + (y - y_1)² +(z - z_1)² = d_1² $
	\item ecuación 2: $(x - x_2)² + (y - y_2)² +(z - z_2)² = d_2² $
	\item ecuación 1: $(x - x_3)² + (y - y_3)² +(z - z_3)² = d_3² $
	\item ecuación 1: $(x - x_4)² + (y - y_4)² +(z - z_4)² = d_4² $
\end{enumerate}

El cuarto punto lo usaremos para comprobar la correci'on de la resolucion o para reemplazar a uno de los anteriores si est'an alineados. En otro caso no es necesario porque sabemos que el punto a localizar est'a dentro del edificio.\bigskip\newline

La resoluci'on anal'itica directa genera denominadores del tipo $x_1 - x_2$, esos denominadores se anulan cuando las antenas est'an en las esquinas de un cuadrado. Esa situaci'on es frecuente asi que es necesario realizar una serie de translaciones y rotaciones para garantizar que los denominadores no se anulan. Hacemos que el origen del sistema de coordenadas est'e en el punto 1, el eje OX pase por el punto 2 y el plano XY pase por el punto 3.  Ver figura \ref{fig:triang1}. 

\begin{figure*}[h!]
	\begin{center}
        		\framebox{\includegraphics[scale=0.6]{Triangular1.png}}
     	\end{center}
    	\caption{Distribuci'on incial de los puntos}\label{fig:triang1}
\end{figure*}


El primer paso es transladar el origen al punto 1 $(x_1,y_1,z_1)$ desde el origen $(0,0,0)$. El vector de translaci'on es $V = (v_1,v_2,v_3) = (x_1,y_1,z_1) - (0, 0, 0)$. Aplicamos la rotaci'on a los tres puntos. Ver figura \ref{fig:triang2}.

\begin{figure*}[h!]
	\begin{center}
        		\framebox{\includegraphics[scale=0.6]{Triangular2.png}}
     	\end{center}
    	\caption{Distribuci'on de los puntos tras la translaci'on}\label{fig:triang2}
\end{figure*}

Rotamos los ejes para transladar el eje OX y que pase por el punto 2.
Ver figura \ref{fig:triang3}.

\begin{figure*}[h!]
	\begin{center}
        		\framebox{\includegraphics[scale=0.6]{Triangular3.png}}
     	\end{center}
    	\caption{'Angulos de rotaci'on}\label{fig:triang3}
\end{figure*}

Es necesario realizar dos rotaciones. La primera rotaci'on se corresponde al 'angulo a: $a = \arccos {x_2 \over {\sqrt{x_2²+z_2^2}}}$. El eje OX se alinea con la proyecci'on del punto 2 sobre el plano XZ. La nuevas coordendas del punto 2 son:

\begin{enumerate}
\item $x_2' = x_2 \sen a + z_2 \cos a$
\item $z_2' = - x_2 \sen a + z_2 \cos a$
\end{enumerate}

La segunda rotaci'on se corresponde al 'angulo b: $b = \arccos {x_2 \over {\sqrt{x_2²+y_2^2}}}$. El eje OX se alinea con el punto 2. La nuevas coordendas del punto 2 son:

\begin{enumerate}
\item $x_2' = x_2 \sen b + y_2 \cos b$
\item $y_2' = - x_2 \sen b + y_2 \cos b$
\end{enumerate}

Las rotaciones tambi'en afectan a los puntos 1 y 2. En el caso del punto 1 no es necesario realizar ninguna operaci'on pues se corresponde al origen. Los resultados para el punto 3 son análogos. \bigskip\newline

Rotamos los ejes para que el plano XY y que pase por el punto 3.
Ver figura \ref{fig:triang5}.

\begin{figure*}[h!]
	\begin{center}
        		\framebox{\includegraphics[scale=0.6]{Triangular5.png}}
     	\end{center}
    	\caption{'Angulos de rotaci'on}\label{fig:triang5}
\end{figure*}

La rotaci'on se corresponde al 'angulo d: $d = \arccos {y_3 \over {\sqrt{z_3²+y_3^2}}}$. El plano XY se alinea con la proyecci'on del punto 3 sobre el plano YZ. 

Tras la resoluci'on anal'itica es necesario dehacer estas modificaciones. 

\begin{enumerate}
\item Dehacemos la tercera rotaci'on -d grados: \bigskip \\
$d = \arccos {y_3 \over {\sqrt{z_3²+y_3^2}}}$
\item Dehacemos la segunda rotaci'on -b grados: \bigskip \\ $b = \arccos {x_2 \over {\sqrt{x_2²+y_2^2}}}$
\item Dehacemos la primera rotaci'on -a grados.\bigskip \\ $a = \arccos {x_2 \over {\sqrt{x_2²+z_2^2}}}$
\item Transladamos -V. \bigskip \\ $V = (x_1,y_1,z_1)$
\end{enumerate}

