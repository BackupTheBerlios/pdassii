La localización corresponde a la resolución del sistema de ecuaciones que se define por el punto de intersección de tres esferas con centro el las coordenadas del punto de acceso y radio la distancia al dispositivo inalámbrico a localizar. Si las punto de acceso son:

\begin{enumerate}
\item Punto 1: 
	\begin{enumerate}
	\item Coordenadas: $(x_1,y_1,z_1)$
	\item Distancia: $d_1$
	\end{enumerate}

\item Punto 2: 
	\begin{enumerate}
	\item Coordenadas: $(x_2,y_2,z_2)$
	\item Distancia: $d_2$
	\end{enumerate}

\item Punto 3: 
	\begin{enumerate}
	\item Coordenadas: $(x_3,y_3,z_3)$
	\item Distancia: $d_3$
	\end{enumerate}

\item Punto 4: 
	\begin{enumerate}
	\item Coordenadas: $(x_4,y_4,z_4)$
	\item Distancia: $d_4$
	\end{enumerate}
\end{enumerate}

Para localizar el punto ($x$,$y$,$z$) tenemos que resolver:
\begin{enumerate}
	\item ecuación 1: $(x - x_1)² + (y - y_1)² +(z - z_1)² = d_1² $
	\item ecuación 2: $(x - x_2)² + (y - y_2)² +(z - z_2)² = d_2² $
	\item ecuación 1: $(x - x_3)² + (y - y_3)² +(z - z_3)² = d_3² $
	\item ecuación 1: $(x - x_4)² + (y - y_4)² +(z - z_4)² = d_4² $
\end{enumerate}

El cuarto punto lo usaremos para comprobar la correción de la resolucion o para reeemplazar a uno de los anteriores si están alineados. En otro caso no es necesario porque sabemos que el punto a localizar está dentro del edificio.\newline

La resolución analítica directa genera denominadreos del tipo $x_1 - x_2$, esos denominadores se anulan cuando las antenas están en las esquinas de un cuadrado. Esa situación es frecuente asi que es necesario realizar un serie de translaciones y rotaciones para garantizar que los denominadores no se anulan. Vamos a garantizar que esto no ocurre haciendo que el origen del sistema de coordenadas esté en el punto 1, el eje OX pase por el punto 2 y el plano XY pase por el punto 3. \newline

\begin{figure*}[h!]
	\begin{center}
        		\framebox{\includegraphics[scale=0.5]{Triangular1.png}}
     	\end{center}
    	\caption{Distribución incial de los puntos}\label{fig:triang1}
\end{figure*}


El primer paso es transladar el origen al punto 1 $(x_1,y_1,z_1)$ desde el origen $(0,0,0)$. El vector de translación es $V = (v_1,v_2,v_3) = (x_1,y_1,z_1) - (0, 0, 0)$. Aplicamos la rotación a los tres puntos. 

\begin{figure*}[h!]
	\begin{center}
        		\framebox{\includegraphics[scale=0.5]{Triangular2.png}}
     	\end{center}
    	\caption{Distribución de los puntos tras la translación}\label{fig:triang2}
\end{figure*}

Rotamos los ejes para transladar el eje OX y que pase por el punto 2.

\begin{figure*}[h!]
	\begin{center}
        		\framebox{\includegraphics[scale=0.5]{Triangular3.png}}
     	\end{center}
    	\caption{Ángulos de rotación}\label{fig:triang3}
\end{figure*}

Es necesario realizar dos rotaciones. La primera rotación se corresponde al ángulo a: $a = \arccos {x_2 \over {\sqrt{x_2²+z_2^2}}}$. El eje OX se alinea con la proyección del punto 2 sobre el plano XZ. La nuevas coordendas del punto 2 son:

\begin{enumerate}
\item $x_2' = x_2 \sen a + z_2 \cos a$
\item $z_2' = - x_2 \sen a + z_2 \cos a$
\end{enumerate}

La segunda rotación se corresponde al ángulo b: $b = \arccos {x_2 \over {\sqrt{x_2²+y_2^2}}}$. El eje OX se alinea con el punto 2. La nuevas coordendas del punto 2 son:

\begin{enumerate}
\item $x_2' = x_2 \sen b + y_2 \cos b$
\item $y_2' = - x_2 \sen b + y_2 \cos b$
\end{enumerate}

Las rotaciones también afectan a los puntos 1 y 2. En el caso del punto 1 no es necesario realizar ninguna operación pues se corresponde al origen. Los resultados pra el punto 3 son análogos. \newline

Rotamos los ejes para que el plano XY y que pase por el punto 3.

\begin{figure*}[h!]
	\begin{center}
        		\framebox{\includegraphics[scale=0.5]{Triangular5.png}}
     	\end{center}
    	\caption{Ángulos de rotación}\label{fig:triang5}
\end{figure*}

La rotación se corresponde al ángulo d: $d = \arccos {y_3 \over {\sqrt{z_3²+y_3^2}}}$. El plano XY se alinea con la proyección del punto 3 sobre el plano YZ. 

Tras la reoslución analítica es necesario dehacer estas modificaciones. 

\begin{enumerate}
\item Dehacemos la tercera rotación -d grados.
\item Dehacemos la segunda rotación -b grados.
\item Dehacemos la primera rotación -a grados.
\item Transladamos -V.
\end{enumerate}

