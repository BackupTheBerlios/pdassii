Al principio de este curso universitario, nuestro grupo se plante'o el reto de desarrollar algo que pudiese resultar 'util para el futuro. Creemos poder decir que lo hemos conseguido. No ya solo el proyecto desarrollado es una buena idea a la hora de implantarlo en lugares estrat'egicos, como puede ser un hospital (y favorecer la tarea de los m'edicos en cuanto a disponibilidad de informaci'on) o un almac'en (de manera que se pudiese llevar un mejor control del inventario y de su plantilla), sino que adem'as el a'no que viene unos compa'neros de la carrera continuarán nuestra idea, dando lugar a nuevas e interesantes perspectivas de cara al futuro. \bigskip \\ Nuestro proyecto, que ha durado un a'no, ha visto su rigurosidad y robustez llevadas a un punto muy alto debido al empe'no puesto durante todo el a'no, y gracias al trabajo desarrollado, hemos logrado una aplicaci'on que tiene un potencial que muy pocos pueden presumir de haber conseguido. Si bien se trata de una versi'on muy ligera de lo que puede llegar a ser, no carece sin embargo de posibilidades e ideas para seguir ampli'andolo. Lo que en un principio parec'ia una iniciativa ardua y complicada, se ha convertido en un trabajo tangible y bien hecho. \bigskip \\ Nuestra idea, ya convertida en aplicaci'on, se propone la dura tarea de localizar un dispositivo port'atil dentro de un recinto gobernado por una red inal'ambrica. Gracias a distintos tipos de algoritmos y m'etodos num'ericos, el sistema es capaz de facilitar todo tipo de ayudas al usuario de ese dispositivo, ya que una vez localizado, las posibilidades son infinitas. Desde mostrar informaci'on de algo cercano, hasta indicarle en cada momento el itinerario a seguir para llegar a un punto lo antes posible, pasando por supuesto por la ayuda profesional que podr'an conseguir tanto m'edicos, como jefes de almacenes o simplemente alguien que desee controlar y facilitar las distintas tareas que puedan afectar a su labor.\bigskip \\ L'ogicamente, esto requiere de un cierto conocimiento del sistema, de sus funcionalidades y sus ventajas para llegar a manejar con soltura las distintas funcionalidades que posee el sistema, pero podemos decir que merece la pena ya que, seg'un se dice, qui'en tiene informaci'on, tiene poder. \bigskip \\ Desde nuestra perspectiva, no queremos sino animar a que distintos tipos de profesionales, desde la rama investigadora hasta aquella donde se desarrollan e implantan las ideas, a que se interesen por involucrarse en una idea que pueda dar mucho de s'i en un futuro no muy lejano, si es que no estamos ya en 'el.