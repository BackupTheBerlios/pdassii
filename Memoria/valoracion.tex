La valoraci'on del proyecto la vamos a realizar en dos 'areas: uso de tecnolog'ias e impacto de la aplicaci'on.\bigskip \\
\begin{itemize}
\item \underline{Uso de tecnolog'ias:}\bigskip \\
En el an'alisis de las tecnolog'ias aplicadas en este proyecto cabe rese'nar el uso de tecnolog'ias punteras como J2ME y redes inal'ambricas. Además está el uso de m'ultiplies tecnolog'ias y su coordinación. \\Integrar bases de datos, servidores de impresi'on, dispositivos inal'ambricos, perif'ericos en red, etc habr'ia sido una tarea imposible si no hubi'eramos llevado a cabo una documentaci'on en inicial no solo sobre las caracter'isticas de las tecnolog'ias y aplicaciones usadas para el desarrollo sino adem'as sobre la compatibilidad de las distinitas tecnolog'ias antes de llevar a cabo el diseño de este proyecto.
\bigskip \\
\item \underline{Impacto de la aplicaci'on}\bigskip \\
Esta aplicaci'on tiene un gran calado no solo por su utilidad actual sino por su adaptabilidad a distintas 'areas.\\ 
En la era de las nuevas tecnolog'ias y los m'oviles 3G una aplicaci'on que facilite las operaciones con archivos de un hospital etc es b'asica. Sus utilidad son muchas, debido a su ampiabilidad, podemos tener una aplicaci'on que no solo envie los informes a la PDA o al dispositivo m'as pr'oximo al usuario, sino que permita ver gr'aficos de evoluci'on de datos del paciente, localizar a otro miembro del personal, reservar el uso de determinas dependencias del centro, consultar el personal que hay de guardia, etc.\\
Adem'as es un diseño reutilizable para otro tipo de actividades:
\begin{itemize}
\item Museos \bigskip \\
En un museo se podr'ia idicar al visitante que camino a de recorrer para llegar a determinada obra, cual es el mejor camino para visitar una serie de obras, informaci'on sobre la 'obra que tenga m'as pr'oxima, localizar al personal del museo para consultas,etc
\item Bibliotecas\bigskip \\
En este caso las utilidades son parecidas. Adem'as se podr'ia, por ejemplo, ubicar los libros en un mapa virtual o consultar sus descripciones antes de desplazarnos a buscarlos. 
\item Almacenes\bigskip \\
Facilitar'ia la tarea a los empleados, que tendr'ian localizados a sus compañeros y podr'ian consultar la ubicaci'on de un paquete o como llegar a el sin encontrarse con los puntos m'as congestionados de gente. Para el encargado le permite conocer el nivel de ocupaci'on del almacen, el estado de las tareas actuales,...
\end{itemize}


\end{itemize}

