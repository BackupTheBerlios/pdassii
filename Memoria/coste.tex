\subsubsection{Coste de la infraestructura}
Definamos primero lo que va a ser la infraestructura:\\ La infraestructura es todo aquello que se debe montar y poner en funcionamiento
para que la herramienta pueda ejecutarse en las mejores condiciones y sin ning'un tipo de problema. Son todos aquellos aparatos, maquinas
y dem'as utensilios que servir'an para la correcta implantaci'on de nuestro sistema, obteniendo as'i un emplazamiento seguro y 
robusto en cuanto a eficacia y posibles problemas.\\ Definamos por tanto que herramientas y aparatos debemos instalar para llevar
a cabo nuestro objetivo, y del mismo modo establezcamos el presupuesto.\\

\begin{itemize}
\item	Puntos de acceso de se'nal inal'abrica.
	\begin{itemize}
	\item Descripci'on : Puntos de acceso inal'abrico con capacidad de transmisi'on de se'nales TCP/IP, tanto en intranet como en internet.
				  Ser'an las encargadas de garantizar una correcta localizaci'on de los dispositivos.
	\item N'umero de unidades : Seg'un lugar de instalaci'on. Normalmente ser'an 4 por planta.
	\item Precio : 450 euros por punto.
	\end{itemize}
\item	Dispositivos tipo PDA para el personal.
	\begin{itemize}
	\item Descripci'on : Son todos aquellos dispositivos que el personal deber'a llevar consigo para poder ser localizados siempre que soliciten un servicio. 
	\item N'umero de unidades : Seg'un plantilla.
	\item Precio : 300 euros por dispositivo.
	\end{itemize}

\item	Dispositivos electr'oicos de diversa 'indole. 
	\begin{itemize}
	\item Descripci'on : Son todos aquellos dispositivos los cuales ser'an utilizados a la hora de resolver las peticiones del usuario.
	Estos pueden ser monitores, impresoras, PDAs, m'oviles etc.
	\item N'umero de unidades : Seg'un estructura del edificio
	\item Precio : Entre 100 y 300 euros por dispositivo.
	\end{itemize}

\end{itemize}

\subsubsection{Coste de desarrollo}
El coste del desarrollo cubre eldesarrollar y adaptar el sistema a la nueva instalaci'on, a las diferentes topolog'ias existentes en el edificio, as'i como el mapeo de cada planta en la base de datos para su futura utilizaci'on. Del mismo modo, el coste de mantenimiento viene inclu'ido aqu'i ya que obviamente, ante futuras situaciones de posible ampliaci'on de sus estructuras, nos vemos obligados a mantener nuestro grado de satisfacci'on del cliente. De este modo, cualquier tipo de obra, construcci'on o modificaci'on en las instalaciones debe ser comunicado de inmediato para su adaptaci'on al c'odigo del programa.
\begin{itemize}
\item Precio : 3000 euros.
\end{itemize}

\subsubsection{Coste de personal}
Este es el coste derivado de la mano de obra proporcionada por nosotros. Es el coste de cada uno de las personas que ha estado 
trabajando en el proyecto.
\begin{itemize}
\item Precio detallado : 45 euros / hora trabajada.
\item Precio : 6750 euros.

\end{itemize}


